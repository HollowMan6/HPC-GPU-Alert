%!TEX root = ../Thesis.tex
\chapter{Code for drawing Figure \ref{fig_load_gpu_histogram_1} \& Figure \ref{fig_load_gpu_histogram_2}}
In Section \ref{sec:algo_method}, we also have the histogram of statistics analysis on windowed GPU load data. Here is the code for drawing those figures:

\begin{lstlisting}[language=Python]
import pandas as pd
import matplotlib.pyplot as plt
plt.rcParams.update({'font.size': 32})
plt.figure(figsize=(14,4))

data = pd.read_csv('load_gpu.csv', usecols=[i for i in range(3, 33)])
new_data = data.transpose()
stat = new_data.describe()
stat.loc['skew'] = new_data.skew()
stat.loc['kurt'] = new_data.kurt()
stat = stat.drop(['count'], axis=0)
stat = stat.transpose()
for kind in ["mean", "std", "min", "25%", "50%", "75%", "max", "skew", "kurt"]:
    pd.cut(stat[kind], bins=100).value_counts().sort_index().to_csv("load_gpu_" + kind + "_histogram.csv")
    figure = stat[kind].plot.hist(bins=100).get_figure()
    figure.savefig("load_gpu_" + kind + "_histogram.pdf")
    figure.clear()

# Verify the threshold selection
def getJobs(name, threshold):
    jobs = []
    for index in stat.index[stat[name] < threshold].tolist():
        jobs.append(data.iat[index, 1])
    jobs.sort()

    return set(jobs)

for job in (getJobs("75%", 20).union(getJobs("max", 32))):
    print(job)
\end{lstlisting}
