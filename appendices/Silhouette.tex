%!TEX root = ../Thesis.tex
\chapter{Code for drawing Figure \ref{fig_silhouette_directly} \& Figure \ref{fig_silhouette_statistics}}
In Section \ref{sec:algo_method}, we have the silhouette analysis on windowed GPU load data. Here is the code for the direct one, as in Figure \ref{fig_silhouette_directly}. Note that for \texttt{load\_gpu.csv}, we have the format as follows, where the data for each entry is the fixed-size (30) sliding window:

\begin{equation*}
hostname,job id,GPU id,data 0,data 1,....,data 28,data 29
\end{equation*}

\begin{lstlisting}[language=Python]
import pandas as pd

data = pd.read_csv('load_gpu.csv', usecols=[i for i in range(3, 33)])
X = data.to_numpy()

import matplotlib.cm as cm
import matplotlib.pyplot as plt
import numpy as np

from sklearn.cluster import KMeans
from sklearn.metrics import silhouette_samples, silhouette_score

range_n_clusters = [2, 3, 4, 5, 6, 7]

plt.rcParams.update({'font.size': 56})

for n_clusters in range_n_clusters:
    print("Start creating plot for n_clusters =", n_clusters)
    # Create a subplot with 1 row and 1 column
    fig, ax1 = plt.subplots(1, 1)
    fig.set_size_inches(18, 18)

    # The 1st subplot is the silhouette plot
    # The silhouette coefficient can range from -1, 1
    ax1.set_xlim([-0.6, 0.4])
    # The (n_clusters+1)*10 is for inserting blank space between silhouette
    # plots of individual clusters, to demarcate them clearly.
    ax1.set_ylim([0, len(X) + (n_clusters + 1) * 10])

    # Initialize the clusterer with n_clusters value and a random generator
    # seed of 10 for reproducibility.
    clusterer = KMeans(n_clusters=n_clusters)
    cluster_labels = clusterer.fit_predict(X)
    print(cluster_labels, cluster_labels.shape)

    # The silhouette_score gives the average value for all the samples.
    # This gives a perspective into the density and separation of the formed
    # clusters
    silhouette_avg = silhouette_score(X, cluster_labels)
    print(
        "For n_clusters =",
        n_clusters,
        "The average silhouette_score is :",
        silhouette_avg,
    )

    # Compute the silhouette scores for each sample
    sample_silhouette_values = silhouette_samples(X, cluster_labels)

    y_lower = 10
    for i in range(n_clusters):
        # Aggregate the silhouette scores for samples belonging to
        # cluster i, and sort them
        ith_cluster_silhouette_values = sample_silhouette_values[cluster_labels == i]

        ith_cluster_silhouette_values.sort()

        size_cluster_i = ith_cluster_silhouette_values.shape[0]
        y_upper = y_lower + size_cluster_i

        color = cm.nipy_spectral(float(i) / n_clusters)
        ax1.fill_betweenx(
            np.arange(y_lower, y_upper),
            0,
            ith_cluster_silhouette_values,
            facecolor=color,
            edgecolor=color,
            alpha=0.7,
        )

        # Label the silhouette plots with their cluster numbers at the middle
        ax1.text(-0.05, y_lower + 0.5 * size_cluster_i, str(i))

        # Compute the new y_lower for next plot
        y_lower = y_upper + 10  # 10 for the 0 samples

    # ax1.set_title("The silhouette plot for the various clusters.")
    ax1.set_xlabel("The silhouette coefficient values")
    ax1.set_ylabel("Cluster label")

    # The vertical line for average silhouette score of all the values
    ax1.axvline(x=silhouette_avg, color="red", linestyle="--")

    ax1.set_yticks([])  # Clear the yaxis labels / ticks
    ax1.set_xticks([-0.6, -0.4, -0.2, 0, 0.1, 0.2, 0.3, 0.4])

    plt.savefig("plot/directly/silhouette_directly_" + str(n_clusters) + ".pdf", format="pdf", bbox_inches="tight")

plt.show()
\end{lstlisting}

For the one with statistics analysis, as in Figure \ref{fig_silhouette_statistics}, the code for drawing it is similar, the only difference is that, we do an aggregation for each entry (fixed-size (30) sliding window), and append those result to the end of each entry, using the nine descriptive statistics metrics, as defined in Subsection \ref{subsec:statistics}: percentiles (25\%, 50\%, 75\%), kurtosis, maximum, minimum, skewness, standard deviation, as well as variance.
