%!TEX root = ../Thesis.tex
\chapter{Code for figures}
In this Chapter, we present the code for the some of figures that are drawn using Python script. These figures are drawn using the Matplotlib plotting library, together with pandas and NumPy.

\section{Figure \ref{fig_silhouette_directly} \& Figure \ref{fig_silhouette_statistics}}
In Section \ref{sec:algo_method}, we have the silhouette analysis on windowed GPU load data. Here is the code for the direct one, as in Figure \ref{fig_silhouette_directly}. Note that for \texttt{load\_gpu.csv}, we have the format as follows, where the data for each entry is the fixed-size (30) sliding window:

\begin{equation*}
hostname,job id,GPU id,data 0,data 1,....,data 28,data 29
\end{equation*}

\begin{lstlisting}[language=Python]
import pandas as pd

data = pd.read_csv('load_gpu.csv', usecols=[i for i in range(3, 33)])
X = data.to_numpy()

import matplotlib.cm as cm
import matplotlib.pyplot as plt
import numpy as np

from sklearn.cluster import KMeans
from sklearn.metrics import silhouette_samples, silhouette_score

range_n_clusters = [2, 3, 4, 5, 6, 7]

plt.rcParams.update({'font.size': 56})

for n_clusters in range_n_clusters:
    print("Start creating plot for n_clusters =", n_clusters)
    # Create a subplot with 1 row and 1 column
    fig, ax1 = plt.subplots(1, 1)
    fig.set_size_inches(18, 18)

    # The 1st subplot is the silhouette plot
    # The silhouette coefficient can range from -1, 1
    ax1.set_xlim([-0.6, 0.4])
    # The (n_clusters+1)*10 is for inserting blank space between silhouette
    # plots of individual clusters, to demarcate them clearly.
    ax1.set_ylim([0, len(X) + (n_clusters + 1) * 10])

    # Initialize the clusterer with n_clusters value and a random generator
    # seed of 10 for reproducibility.
    clusterer = KMeans(n_clusters=n_clusters)
    cluster_labels = clusterer.fit_predict(X)
    print(cluster_labels, cluster_labels.shape)

    # The silhouette_score gives the average value for all the samples.
    # This gives a perspective into the density and separation of the formed
    # clusters
    silhouette_avg = silhouette_score(X, cluster_labels)
    print(
        "For n_clusters =",
        n_clusters,
        "The average silhouette_score is :",
        silhouette_avg,
    )

    # Compute the silhouette scores for each sample
    sample_silhouette_values = silhouette_samples(X, cluster_labels)

    y_lower = 10
    for i in range(n_clusters):
        # Aggregate the silhouette scores for samples belonging to
        # cluster i, and sort them
        ith_cluster_silhouette_values = sample_silhouette_values[cluster_labels == i]

        ith_cluster_silhouette_values.sort()

        size_cluster_i = ith_cluster_silhouette_values.shape[0]
        y_upper = y_lower + size_cluster_i

        color = cm.nipy_spectral(float(i) / n_clusters)
        ax1.fill_betweenx(
            np.arange(y_lower, y_upper),
            0,
            ith_cluster_silhouette_values,
            facecolor=color,
            edgecolor=color,
            alpha=0.7,
        )

        # Label the silhouette plots with their cluster numbers at the middle
        ax1.text(-0.05, y_lower + 0.5 * size_cluster_i, str(i))

        # Compute the new y_lower for next plot
        y_lower = y_upper + 10  # 10 for the 0 samples

    # ax1.set_title("The silhouette plot for the various clusters.")
    ax1.set_xlabel("The silhouette coefficient values")
    ax1.set_ylabel("Cluster label")

    # The vertical line for average silhouette score of all the values
    ax1.axvline(x=silhouette_avg, color="red", linestyle="--")

    ax1.set_yticks([])  # Clear the yaxis labels / ticks
    ax1.set_xticks([-0.6, -0.4, -0.2, 0, 0.1, 0.2, 0.3, 0.4])

    plt.savefig("plot/directly/silhouette_directly_" + str(n_clusters) + ".pdf", format="pdf", bbox_inches="tight")

plt.show()
\end{lstlisting}

For the one with statistics analysis, as in Figure \ref{fig_silhouette_statistics}, the code for drawing it is similar, the only difference is that, we do an aggregation for each entry (fixed-size (30) sliding window), and append those results to the end of each entry, using the nine descriptive statistics metrics, as defined in Subsection \ref{subsec:statistics}: percentiles (25\%, 50\%, 75\%), kurtosis, maximum, minimum, skewness, standard deviation, as well as the variance.

\clearpage

\section{Figure \ref{fig_load_gpu_histogram_1} \& Figure \ref{fig_load_gpu_histogram_2}}
In Section \ref{sec:algo_method}, we also have the histogram of statistics analysis on windowed GPU load data. Here is the code for drawing those figures:

\begin{lstlisting}[language=Python]
import pandas as pd
import matplotlib.pyplot as plt
plt.rcParams.update({'font.size': 32})
plt.figure(figsize=(14,4))

data = pd.read_csv('load_gpu.csv', usecols=[i for i in range(3, 33)])
new_data = data.transpose()
stat = new_data.describe()
stat.loc['skew'] = new_data.skew()
stat.loc['kurt'] = new_data.kurt()
stat = stat.drop(['count'], axis=0)
stat = stat.transpose()
for kind in ["mean", "std", "min", "25%", "50%", "75%", "max", "skew", "kurt"]:
    pd.cut(stat[kind], bins=100).value_counts().sort_index().to_csv("load_gpu_" + kind + "_histogram.csv")
    figure = stat[kind].plot.hist(bins=100).get_figure()
    figure.savefig("load_gpu_" + kind + "_histogram.pdf")
    figure.clear()

# Verify the threshold selection
def getJobs(name, threshold):
    jobs = []
    for index in stat.index[stat[name] < threshold].tolist():
        jobs.append(data.iat[index, 1])
    jobs.sort()

    return set(jobs)

for job in (getJobs("75%", 20).union(getJobs("max", 32))):
    print(job)
\end{lstlisting}

\section{Figure \ref{fig_benchmark}}
In Subsection \ref{subsec:performance}, we have a performance benchmark, here is the code for drawing the related Figure \ref{fig_benchmark}, you can also find the raw data for the benchmark result in the code below:

\begin{lstlisting}[language=Python]
import matplotlib.pyplot as plt
import numpy as np

# Data
methods = ["Polling CAGG", "Polling Direct Query", "Trigger CAGG", "Trigger in Memory"]
colors = ["#36A2EB", "#FF6384", "#FF9F40", "#9966FF"]
gpu_write_delay = [
    [0.321007643, 0.660196607, 1.279614867, 2.433749516, 5.131623884, 11.624568741],
    [0.31859915, 0.625456453, 1.191235464, 2.350517039, 4.958478354, 9.977873813],
    [0.765987412, 1.609155407, 2.73052539, 5.176909349, 10.908855175, 20.96940738],
    [0.65528452, 1.285481557, 2.53125466, 5.007170087, 9.824896244, 19.604722893]
]
read_iteration = [
    [1.201504459, 2.102009134, 3.58900885, 7.282948943, 13.567867101, 18.581255946],
    [2.245655323, 3.443534234, 6.191009134, 12.676023435, 26.139007435, 55.204802484],
    [13.184558, 30.957367, 56.111442, 101.592729, 212.071778, 433.089509],
    [12.250112, 30.117734, 52.671703, 91.90153, 188.53631, 388.158221]
]
index = range(len(gpu_write_delay[0]))

# Plot for GPU Write Delay
fig, axes = plt.subplots(3, 1, figsize=(8, 12))
for i, method in enumerate(methods):
    axes[0].bar([x + i * 0.2 for x in index], gpu_write_delay[i], 0.2, label=method, color=colors[i])
    axes[0].plot([x + i * 0.2 for x in index], gpu_write_delay[i], color=colors[i], marker='o', linewidth=1, markersize=2)
    # Add horizontal lines to the first plot
    for tick in gpu_write_delay[i]:
        axes[0].axhline(y=tick, color='grey', linestyle='dotted', linewidth=0.3)
axes[0].set_xlabel('Node Count * GPU')
axes[0].set_ylabel('Time (s)')
axes[0].set_title('Delay in Database Writing Transaction Commit')
axes[0].set_yscale('log')
axes[0].set_xticks([x + 0.2 * 3 / 2 for x in index])
axes[0].set_xticklabels(['93*8', '186*8', '372*8', '744*8', '1489*8', '2978*8'])
axes[0].legend()
y_ticks_0 = [0.5, 1, 2, 3, 4, 5, 6, 7, 9, 11, 14, 17, 20]
axes[0].set_yticks(y_ticks_0)
axes[0].set_yticklabels([str(tick) for tick in y_ticks_0])

# Plot for Read Iteration
for i in range(2):
    method = methods[i]
    axes[1].bar([x + i * 0.2 for x in index], read_iteration[i], 0.2, label=method, color=colors[i])
    axes[1].plot([x + i * 0.2 for x in index], read_iteration[i], color=colors[i], marker='o', linewidth=1, markersize=2)
    # Add horizontal lines to the second plot
    for tick in read_iteration[i]:
        axes[1].axhline(y=tick, color='grey', linestyle='dotted', linewidth=0.3)
axes[1].set_xlabel('Node Count * GPU')
axes[1].set_ylabel('Time (s)')
axes[1].set_title('Time Needed in Completing a Read Loop Iteration')
axes[1].set_yscale('log')
axes[1].set_xticks([x + 0.2 * 3 / 2 for x in index])
axes[1].set_xticklabels(['93*8', '186*8', '372*8', '744*8', '1489*8', '2978*8'])
axes[1].legend()
y_ticks_1 = [0.5, 1, 2, 2.5, 3, 4, 5, 6, 7, 9, 12, 15, 20, 25, 30, 40, 50, 60]
axes[1].set_yticks(y_ticks_1)
axes[1].set_yticklabels([str(tick) for tick in y_ticks_1])

# Plot for Memory Status Dump
for i in range(2, 4):
    method = methods[i]
    axes[2].bar([x + i * 0.2 for x in index], read_iteration[i], 0.2, label=method, color=colors[i])
    axes[2].plot([x + i * 0.2 for x in index], read_iteration[i], color=colors[i], marker='o', linewidth=1, markersize=2)
    # Add horizontal lines to the third plot
    for tick in read_iteration[i]:
        axes[2].axhline(y=tick, color='grey', linestyle='dotted', linewidth=0.3)
axes[2].set_xlabel('Node Count * GPU')
axes[2].set_ylabel('Time (ms)')
axes[2].set_title('Delay in Completing a Memory Status Dump')
axes[2].set_yscale('log')
axes[2].set_xticks([x + 0.2 * 3 / 2 for x in index])
axes[2].set_xticklabels(['93*8', '186*8', '372*8', '744*8', '1489*8', '2978*8'])
axes[2].legend()
y_ticks_2 = [5, 10, 20, 25, 30, 40, 50, 60, 70, 90, 120, 150, 200, 250, 300, 400, 500]
axes[2].set_yticks(y_ticks_2)
axes[2].set_yticklabels([str(tick) for tick in y_ticks_2])
plt.tight_layout()
plt.savefig("benchmark-data.pdf")
\end{lstlisting}
